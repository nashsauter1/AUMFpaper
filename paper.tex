\documentclass[12pt]{article}
\title{Impacts of the AUMF Resolutions on Obama and Trump Administration Foreign Policy}
\author{Nash Sauter}
\date{Fall Semester, POLS 1600}

\usepackage{setspace} 
\usepackage[utf8]{inputenc} 
\usepackage{CormorantGaramond}
\usepackage[hidelinks]{hyperref}
\usepackage{fancyhdr}
\usepackage{lastpage}
\usepackage{microtype}
\usepackage{blindtext}
\usepackage{indentfirst}
\usepackage[margin=1in]{geometry}
\usepackage[american]{babel}
\usepackage{csquotes}
\usepackage[authordate,backend=biber]{biblatex-chicago}
\usepackage{url}
\addbibresource{aumf.bib}
\pagestyle{fancy}
\fancyhf{}
\rhead{Sauter \thepage}
\setlength{\headheight}{15pt}
\linespread{2}
\raggedright{}
\setlength\parindent{0.5in}
\renewcommand{\headrulewidth}{0pt}
\addtolength{\skip\footins}{2pc plus 5pt}
\urlstyle{same}

\begin{document}
\maketitle
\pagenumbering{gobble}
\newpage
\pagenumbering{arabic}

\section*{Introduction}
\subsection*{Background}
In the days following the terrorist attacks against the United States on September 11th, 2001, both houses of Congress took to action and passed a bipartisan resolution titled S.J.Res.23: Authorization for Use of Military Force.
As this joint resolution was created to address an urgent problem, being the national security of the United States, the resolution was passed through both chambers of Congress on September 14th and signed into law by President George W. Bush only four days later the on September 18th.
Although the speed at which both parties came together to take action was a monumental achievement, this short time frame led to the resolution giving the President of the United States a broad set of war powers instead of any specifically defined actions.
In the text of the resolution, it states that Congress will give the President the authority to ``use all necessary and appropriate force against those nations, organizations, or persons he determines planned, authorized, committed, or aided the terrorist attacks that occurred on September 11, 2001, or harbored such organizations or persons, in order to prevent any future acts of international terrorism against the United States by such nations, organizations or persons. \autocite{daschle2001}''
At the time, this vague language and broad power was necessary since the threat was still being assessed and no one knew of an exact solution to the instability, damage, and national paranoia caused by the attacks.
However, as time has gone on, the Presidents after George W. Bush have interpreted both this resolution, as well as the 2002 AUMF and 1991 AUMF resolutions to a lesser extent, in a wide variety of ways that go beyond the scope of the original intentions of the resolutions.
This paper will operate under the following question: how have the AUMF resolutions shaped foreign policy approaches under the Obama and Trump administrations?

The purpose of this paper is to investigate how the passing of the 2001 AUMF resolution as well as the various judicial, legislative, and executive interpretations of those resolutions has shaped the executive branch foreign policy decisions and international relations approaches of the Obama and Trump administrations.
First, this paper will review the history and original intents of the 2001 AUMF resolution. 
Since this paper will put the most focus into the 2001 AUMF, the 1991 AUMF and 2002 AUMF will only be briefly mentioned as they have primarily been invoked alongside the 2001 AUMF resolution as alternative justification rather than being especially distinct in their impacts.
Next, this paper will describe some of the various legal interpretations of the 2001 AUMF resolution in order to set a framework to look at the executive action through.
Then, this paper will look their uses under the Obama administration as well as the various precedents set by those actions.
Finally, it will examine the ways they've been used under the present-day Trump administration in order to see if their uses have changed or stayed consistent with the Obama administration precedents.
The Obama administration section will most likely take up the bulk on the analysis as it will have the most room for interpretation, and the literature regarding Trump administration AUMF is much more limited.

\subsection*{Research Process}
Throughout this paper, there will be references to a variety of academic articles,  along with one book and a few other supplementary sources such as speech transcripts from the Congressional Record and news articles.
I have designated four of these sources to be my major sources, meaning that they will be referenced the most throughout this paper compared to any of the other sources present.
There are two main criteria that I chose in order to determine which sources I would be giving the most weight to.
First, they are the most relevant pieces of literature I could find for the subject that this paper is addressing.
They all deal directly with the AUMF resolutions and provide many frameworks, historical accounts, arguments, and analyses that will help with answering the research question of this paper.
Second, they address a wide range of topics within the research subject.
While they all deal with similar questions, some of them focus more on the legal interpretations, some focus on the institutional process that led to those interpretations, and others look at the direct, practical impacts of the resolutions.
They also cover time periods all the way from the beginning of the formation of the resolutions to the end of the Obama administration and beginning of the Trump administration.

The most important source for this research paper will be the book \textit{Counter-Terrorism from the Obama Administration to President Trump: Caught in the Fait Accompli War} by Donna Starr-Deelen \autocite{starr-deelen2018}.
Starr-Deelen has taught Public International Law and Human Rights Law and has written several books regarding the evolving nature and role of terrorism across many presidential administrations.
This book primarily focuses on the \textit{fait accompli} war left behind by George W. Bush, meaning a war that was decided on before those affected could hear about it.
This \textit{fait accompli} was less on the grounds of the actual forces and conflicts of the war, but instead was the ideological war on terror caused by the ``virtually irreversible policy choices President Bush made regarding the use of force and targeting, detention, and interrogation methods.'' \autocite[2]{starr-deelen2018}.
The book looks at how these irreversible choices affected the foreign policy approaches of the Obama administration as well as the first few months of the Trump administration.
Although this book covers a much wider range of topics than just the impacts of the AUMF resolutions, it provides many historical accounts and political theories that will be instrumental in the analysis of the subject that will be present throughout this paper.

Another source that will be important for this research paper is \textit{2001 Authorization for Use of Military Force: Issues Concerning Its Continued Application} by Matthew Weed \autocite{weed2015}, a nonpartisan specialist in foreign policy legislation for the Congressional Research Service.
This paper covers the initial legislative background behind the 2001 AUMF, and also has a lot of information about the specific policies enacted under the Obama administration that were legally justified using the AUMF resolutions, and the differences between what President Obama has said about the AUMF resolutions from the actions he took.
This will be a very useful account in order to understand what events have actually happened before going into any analysis on the subject.
This source provides also provides background on the processes that went into drafting and implementing the 2001 AUMF\@.

The next major source that will be used in this paper is \textit{The Contemporary Presidency: Stretching the 2001 AUMF} by Shoon Kathleen Murray \autocite{murray2015}, a PhD political scientist that focuses on foreign policy and has done a variety of works on the AUMF resolutions.
This source deals with both the Bush and Obama presidencies, but will mostly be used for the section of this paper about historical background.
The article has a good amount of analysis looking at quotes and personal accounts of the legislators that were involved with the drafting, negotiation, and passage of the AUMF as well as similar accounts for the Bush administration lawyers.
These are used to create detailed accounts of the intentions of both Congress and the Bush administration.
Finally, the source \textit{Obama's AUMF Legacy} by Curtis A. Bradley and Jack Landman Goldsmith \autocite{bradley2016}, researchers for the Duke University of Law and Harvard Law respectively, will be used to bring the Obama administration AUMF interpretation to a more international scale.
The paper looks at the impacts through the events regarding Guantanamo Bay as well as the relation of the AUMF resolutions to international law.

\section*{Origins of the 2001 Bush Administration AUMF}
\subsection*{The Intent of the Bush Administration}
Following the terrorist attacks against the United States that took place on September 11th, 2001, President George W. Bush's role as the acting Commander-in-Chief and spokesman for the United States was tasked with both identifying the perpetrators of the attacks as well as deciding what counter-measures to take as a result.
President Bush identified two main groups as the parties responsible for the attacks: Al-Qaeda and the Taliban.
It was believed that Al-Qaeda was the group that planned and initiated the attacks, while the Afghani Taliban were also guilty due to their harboring of Al-Qaeda members within their territory.
With the purposes of providing an immediate response that would effectively utilize Congress's constitutional war powers, the 107th Congress passed an Authorization for Use of Military Force (AUMF) resolution, similar to the one passed in 1991 under the George H.W. Bush administration.
This resolution was passed through both the Senate and House in identical form on September 14th, just three days after the attacks \autocite{zotero-165}.
Although that length of time is notably fast compared to many other landmark pieces of legislation passed through Congress, there was still a lot of discussion between the legislators representing both of the political parties, leaders in the Bush administration, as well as legal and foreign policy experts.

Prior to 2001, the United States government had generally treated counterterrorism efforts as legal and law enforcement issues rather than one of military and war powers.
The response to the 9/11 attacks by both the American public and government officials was different, as it seemed more like an planned attack on United States soil by a centralized group rather than a decentralized and isolated event.
Additionally, there was also a much larger loss of domestic life compared to previous attacks, causing more tension among the people of the United States.
On the day following the attacks, in an address to the American people, President Bush described the events as such:

\begin{displayquote}
``The deliberate and deadly attacks, which were carried out yesterday against our country, were more than acts of terror. They were acts of war. This will require our country to unite in steadfast determination and resolve. Freedom and democracy are under attack. The American people need to know we're facing a different enemy than we have ever faced. This enemy hides in shadows and has no regard for human life. This is an enemy who preys on innocent and unsuspecting people, then runs for cover, but it won't be able to run for cover forever. This is an enemy that tries to hide, but it won't be able to hid forever. This is an enemy that thinks its harbors are safe, but they won't be safe forever. This enemy attacked not just our people but all freedom-loving people everywhere in the world.'' \autocite{bush2001}
\end{displayquote}

\noindent
By doing this, President Bush framed the attacks as more than just a tragic occurrence, they were a direct declaration of war on the free world.
In response to Al-Qaeda's act of war, the Bush administration decided to push for a legal authorization for a war against the 9/11 perpetrators.
Vice President Dick Cheney consulted with his lawyers along with members of the White House Counsel and the Department of Justice Legal Counsel to see what kind of authority would be needed for this.
Following this, President Bush met with congressional leaders to specifically request an resolution giving an authorization for military force to the executive branch.
Additionally, Bush also had Vice President Cheney's team of counsel members create a first draft of the resolution \autocite[176]{murray2015}.
This draft was considered by Senate Majority Leader Tom Daschle, who would later be the sponsor for the 2001 AUMF resolution, as a blank check that would allow the Bush administration to use any amount of military force without consequence.
Congress decided to pass an AUMF under the conditions that it would limited to those responsible for the attacks, as well as that it would remain consistent with the 1973 War Powers Resolution.
It was also said that the Bush administration tried to add a clause to the draft resolution that would allow for military action \emph{within} the United States in the final moments before the vote, but it was ultimately rejected by Congress.
After the members of Congress agreed to a set of conditions along with the wording of the resolution, they passed the resolution through both the House and Senate in approximately 28 minutes with only a single vote in dissent \autocite[177]{murray2015}.

\subsection*{The Intent of Congress}
Although Congress went through lengthy discussions in order to make sure the wording of the resolution wouldn't have unforeseen consequences, there were still some parts that ending up being broader than intended.
The 2001 AUMF authorized the President to have access to Necessary and Proper clause in Article 1 of the Constitution in order to both combat and detain the perpetrators of the 9/11 attacks, and as such gave President Bush a large amount of military power that would typically be in the hands of Congress.
The first example is the phrase ``organizations, or persons'' which can be found in the first operative clause of the resolution.
This was dissimilar from Congress's historical declarations of war as well as the 1991 Authorization for Use of Military Force Against Iraq Resolution, which not only specifically chose the government of Iraq as a target, but also was much more clear in its goals.
Additionally, the 2001 AUMF resolution was somewhat broad in the way it described which non-state actors and/or nations should be targeted.
Malcolm Brooks Savage III describes how there was only one limitation placed on the powers of the President in the operational clauses of the 2001 AUMF resolution, being the 9/11 nexus requirement, which attempts to define the targets of the authorization as only being those who were involved in the 9/11 attacks.
The first operative clause states that the President can use force against all ``nations, organizations, or persons \emph{he determines} planned, authorized, committed, or aided the terrorist attacks.''
The phrase ``he determines'' is especially notable as it could presumably allow the President to target any group that was remotely connected to the attacks, even if they are outside of the original Congressional intent of the resolution \autocite[5]{savage2015}.

The shift of military focus from state to non-state actors marked a drastic change within the foreign policy of the United States.
Although the Bush administration specifically identified Al-Qaeda and the Taliban as the targets of their response plans, the language of the 2001 AUMF resolution gave additional power to the President that went beyond those two groups specifically.
Additionally, the phrasing led to interpretations that would allow military force to be exerted not just on group actors, but also individuals.
As a result of this, President Bush, as well as all future presidents that chose to implement the 2001 AUMF into their foreign policy decisions, was able to choose targets in both a wider scope due to the lack of specificity and increased executive discretion, as well as choose targets that were much more specific, such as specific people that were deemed as terrorists \autocite[3]{weed2015}.
As will be discussed later in this paper, these parts of the 2001 AUMF, regardless of whether they should be considered features or loopholes, have had many direct consequences on the foreign policy of the United States for the last two decades.

It is worth noting that there is strong evidence that the broad discretion that can be found in the operative clauses of the 2001 AUMF resolution are the result of a short time frame and unforeseen consequences rather than purposeful intent from Congress to give the President a blank check of military force.
Senator Robert Byrd said the following in a Congressional meeting on October 1st, 2001:

\begin{displayquote}
``I began to have some qualms over how broad a grant of authority Congress
gave him in our rush to act quickly.
Because of the speed with which it was passed, there was little discussion establishing a foundation for the resolution.
Because of the paucity of debate, it would be difficult to glean from the
record the specific intent of Congress in approving S.J. Res. 23.
There were after-the-fact statements made in the Senate, and there was some debate in the House, but there was not the normal level of discussion or the normal
level of analysis of the language prior to the vote that we have come to expect in the Senate \autocite[S9949]{byrd2001a}.''
\end{displayquote}

\noindent
This proves how despite the level of deliberation taken, it was still rushed far beyond the amount of time expected of such an important piece of legislation.
Although there was supposedly some benefit to this, being that the situation at hand demanded immediate action, this gives evidence to the idea that the text of the resolution wasn't completely aligned with the interests or intents of Congress.

\begin{displayquote}
``Two aspects of the resolution are key: First, the use of force authority granted to the President extends only to the perpetrators of the September 11 attack.
It was not the intent of Congress to give the President unbridled authority—I hope it wasn’t to wage war against terrorism writ large without the advice and consent of Congress.
That intent was made clear when Senators modified the text of the resolution proposed by the White House to limit the grant of authority to the September 11 attack \autocite[S9949]{byrd2001a}.''
\end{displayquote}

\noindent
This quote describes the intent behind the 9/11 nexus requirement that was discussed by Savage.
Later in the speech, Senator Byrd elaborates on the resolution draft that was proposed by the White House.
This draft, in addition to all of the powers granted in the version of the resolution which was later passed by Congress, would have allowed for the President to ``deter and pre-empt any future acts of terrorism or aggression against the United States \autocite[S9949]{byrd2001a}.''
The rejection of this proposed clause proves that there was a congressional intent to limit the scope of the authorization to some degree.

Beyond this, the resolution also aimed to further the restrictions on the power of the executive branch by invoking the War Powers Resolution, which was passed under President Nixon in 1973 \autocite{zotero-176}.
Section (2) (b) (1) of the resolution requires that specific statutory authorization should follow the meaning of section (5) (b) of the War Powers Resolution \autocite{daschle2001}.
The War Powers Resolution requires in section (4) (a) (1) that the President must submit a report to Congress within 48 hours following any case involved the United States Armed Forces.
The report must include information such as circumstances, legal authority, and the estimated scope of the involvement.
The section referenced in the 2001 AUMF resolution, section (5) (b), is the section that requires the President to get specific authorization for their actions from Congress within a 60 day period after submitting the report.
This authorization can either come in the form of a declaration of war or a specific authorization the use the Armed Forces, and can only be waived if Congress grants an extension or is physically unable to meet \autocite{zablocki1973}.
The 2001 AUMF resolution also references the War Powers Resolution in section (2) (b) (2) by stating that ``nothing in this resolution supercedes any requirement of the War Powers Resolution.''
Senator Byrd comments on these invocations of the War Powers Resolution by saying that ``extended operations against other parties or nations not involved in the attack would require —or would it— additional specific authorization beyond the 60 day period provided for in the War Powers Resolution. Whether the language of S.J. Res. 23 adequately supports the intent is another matter \autocite[SS950]{byrd2001a}.''

\section*{The Obama Era}
\subsection*{Initial Choices and Theories}
Following the end of the Bush administration and the beginning of the Obama administration, the war on terror was passed directly in President Obama's hands.
Compared to prior administrations of Reagan, Clinton, and Bush, President Obama had much less freedom in regards to choosing how to operate American foreign policy as the Commander-in-Chief.
As described by Starr-Deelen, the Bush administration's unique approach to counterterrorism left behind a \textit{fait accompli} war, meaning that the consequences of the war would largely fall under the Obama administration even though President Obama had no choice but to accept them.
The main example of this is how President Bush shifted the United States away from the law enforcement approach to counterterrorism in favor of the armed conflict approach over the course of his presidency \autocite[2]{starr-deelen2018}.
The law enforcement paradigm, which had historically been favored prior to the Bush administration, seeks to treat terrorism as a criminal act rather than a national security issue.
Supporters of this approach believe that terrorism is distinct from other violent acts such as war or invasion.
Counterterrorism tends to be more focused and centralized under this paradigm, as individual terrorists or specific terrorist groups are to be targeted by policymakers on a case-by-case basis.
The policy focus also tends to put more of a focus on domestic protection and preemption rather than international military force.
Additionally, the law enforcement model would place more of an emphasis on law enforcement mechanisms rather than military force.
This would optimally result in terrorists being apprehended by police forces and then would seek to implement criminal trials with prosecutors and civilian juries.
Through this emphasis on legal systems, it would also seek to follow principles of due process, just like any other severe criminal act \autocite[3]{starr-deelen2018}.

The main alternative to the law enforcement paradigm would be the armed conflict approach.
This paradigm treats counterterrorism as a national security threat rather than a criminal act.
Supporters believe that the existence of terrorist acts is more of an attack on the state apparatus than an attack on people.
Under this approach, terrorists should be fought and apprehended by military forces by any means, notably including lethal force.
When captured, the approach is also much more willing to waive principles of due process through the use of military courts and petitions.
As a result of this, the paradigm is also known as the ``war on terror'' approach.
Counterterrorism under this paradigm tends to follow a more international and aggressive approach where it tries to directly combat and destroy terrorism regardless of geographical or temporal boundaries.
In regards to framing, the approach tends to contrast from the law enforcement approach by labeling terrorists as enemy combatants rather than criminals.
The Bush administration was a strong proponent of the armed conflict approach, as it labeled jihadists in Al-Qaeda and the Taliban as soldiers in an ideological and violent war against the United States and the broader West \autocite[3]{starr-deelen2018}.

In response to the war on terror approach, the newly formed Obama administration had three options as an alternative to the preservation of the status quo.
First, President Obama could embrace the armed conflict paradigm, but announce the end of the war on terror.
This approach would attempt to simply move on from the Bush Era and create a somewhat of clean slate.
This could either result in a decrease in foreign interventions or could result in a formation of a new war on terror.
Second, President Obama could take a more aggressive approach against the Bush administration's approach by renouncing the war on terror and taking action against the legal violations of the Bush administration.
This approach would seek to likely seek to return back to the prior status quo of law enforcement-based counterterrorism.
Finally, President Obama could attempt to create a third path by combining aspects of both the law enforcement and armed conflict paradigms.
President Obama would end up deciding to take the third approach that would maintain the fight against the terrorism while simultaneously aiming to reduce executive overreach and returning to a constitutional balance of power.
Upon taking office, President Obama stated that the President does not have the power to ``unilaterally authorize a military attack in a situation that does not involve stopping an actual or imminent threat to the nation \autocite[6]{starr-deelen2018}.''
He also signed executive orders to close the Guantanamo Bay prison, end the use of torture in interrogations, and legally review prior detention policies from the Bush administration.
President Obama's deputy nation security adviser also echoed this by planning to implement a model that would incorporate both the criminal justice system as well as international military force.
This new approach would also seek to shift the focus toward combating the conditions behind extremism rather than the extremist terrorists themselves \autocite[7]{starr-deelen2018}.

\subsection*{Expansions of the AUMF Scope}
Although this third way policy was implemented to a certain degree, the Obama administration's use of the 2001 AUMF resolution involved stretching its application far beyond the original Congressional intent.
Specifically, the authorization started to be applied to groups and conflicts far beyond those directly involved in the 9/11 terrorist attacks.
The primary way that this was done was by expanding the target of executive force to groups that could be associated with the Afghanistan-based Al-Qaeda.
By designating groups as \emph{co-belligerents} with the Afghani Al-Qaeda, the Obama administration was effectively able to designate groups that weren't involved in the 9/11 attacks as being covered by the 9/11 nexus.
This could even include groups that had not been formed until after 2001.
During President Obama's first term, the administration first expanded their targeting to Al-Qaeda in the Arabian Peninsula (AQAP), which was primarily located in Yemen.
Additionally, other non-Al-Qaeda groups such as al-Shabaab in Somalia and the Khorasan Group in Syria were also most likely considered by the Obama administration to be targets covered by the 2001 AUMF \autocite[189]{murray2015}.

Part of this expansion in executive power was an corresponding expansion of ambiguity and secrecy around Obama administration's military actions.
The drone strike program under the Obama administration wasn't officially acknowledged for three years despite estimates that more strikes took place in Pakistan during President Obama's first year in office than the amount under Bush's previous seven years in office combined \autocite[189]{murray2015}.
This included a secret legal decision in the Obama administration's Office of Legal Counsel in 2010 which allowed for the targeted killing of American citizens abroad as cobelligerents. 
This resulted in the drone killing of Anwar Awlaki, a US citizen that was associated with AQAP \autocite[191]{murray2015}.
The marked a shift back to the armed conflict approach of the Bush administration, as it became clear that lethal force and a lack of due process could be used in the name of counterterrorism at the discretion of the executive branch.

To achieve these expansions, the Obama administration regularly cited the 2001 AUMF as a legal basis.
They have further claimed that the 2001 AUMF resolution has no limits in regards to geography or temporal scale \autocite[108]{boyle2015}.
This claim, if followed, would lead to a scope of power that goes far beyond that of the original authors of the 2001 AUMF resolution.
In regards to the time scale, it would be reasonable to assume that this interpretation would leave to the power of the authorization in the hands of the president until Al-Qaeda is completely defeated.
Due the group's expansion around the world, Obama administration officials have estimated that the war on terror could end up lasting for upwards of 30 years.
Additionally, the application of the 2001 AUMF could extend beyond Al-Qaeda, as it was commonly used to justify military action against ISIS in Iraq and Syria despite the fact that ISIS did not exist during the 9/11 attacks and also has a ``remarkably loose affiliation'' with Al-Qaeda \autocite[111]{boyle2015}
The interpretation would also give legal legitimacy to targeted attacks in any geographic location where links could be made to Al-Qaeda or the Taliban.
There is reason to believe that this was used to justify Obama administration attacks not just in other Middle Eastern nations such as Iraq and Libya, but also in other regions in Asia such as Indonesia and the Philippines.
It is unknown exactly where the Obama administration has invoked the 2001 AUMF as many attacks were either unrecognized or done under private criteria that was set by the administration \autocite[109]{boyle2015}.

\subsection*{The Preservation of Executive Detention Policy}
Although President Obama remained in favor of closing the Guantanamo Bay detention center throughout his time in office, the actions of the Obama administration legal team suggest that he was still in favor of continuing to detain prisoners at the site without civil trial, albeit with some different requirements from those under President Bush.
Most of this position can be explained with a memorandum known as the ``Obama Memo'' in which parts of the Bush administration's position on the detention of suspected and known terrorists were preserved, refuted, and adjusted.
The main way that the memo differed from the Bush administration's stance was by removing any mention to Article II of the Constitution, which was previously interpreted by the Bush administration as giving the President an inherent right to unilateral detention power.
This shift would presumably give some of the power of detention back to Congress, and would allow for Congress to clarify whether or not an authorization for force would include detention authority \autocite[44]{brill2010}.
The Obama Memo also seemed to invoke a further care for international law, as it contradicted the Bush administration's arguments that international treaties such as the Geneva Conventions were limited and didn't apply to their actions.
Specifically, the memo referenced the United Nations Charter, NATO, the OAS, the Geneva Conventions, and the International Committee of the Red Cross \autocite[46]{brill2010}.
The result of these changes included the limiting the power of the executive branch by providing oversight from both the legislative branch as well as the international community.
These changes were presumably meant to aid in the eventual closing of Guantanamo detention facilities as they would restrict the ability to bring new groups of detainees into the scope of executive detention power.

Despite these proposals, the Obama administration remained remarkably similar in policy to the Bush administration when it came to the detainees that were already present in the facilities during President Obama's two terms in office.
First, the Obama Memo still used an interpretation of the 2001 AUMF as a legal justification for detaining the prisoners.
In the section regarding who could be detained by the executive branch, the memo states that the President can detain people who ``were part of, or substantially supported, Taliban or al Qaeda forces or associated forces.''
This was virtually identical to the specifications of the Bush administration apart from the word ``substantially''.
The inclusion of that word likely provided no changes to the policy of detention as there were no definitions or clarifications for what that word would mean in a legal sense \autocite[46]{brill2010}.

Additionally, there is reason to believe that the changes to include international law and decision-makers in the detention process were mostly aesthetic.
While the administration spoke to the relevance of international law in the interpretation and application of the 2001 AUMF, their actual interpretations of the law generally increased presidential discretion even when it was at the cost of reduced international oversight.
In some cases, the Obama administration even attempted to invoke international law as a reason to increase unilateral executive power.

\begin{displayquote}
``On many of the most controversial issues of presidential power under the AUMF — including the legality of indefinite military detention of persons captured away from a traditional battlefield, the targeting and capturing of alleged terrorists who are not formal members of an identified enemy group, and the use of force in nations with which the United States is not at war — the administration justified its actions by reference to international law \autocite[639]{bradley2016}.''
\end{displayquote}

\noindent
In addition, the Obama administration argued that the 2001 AUMF granted the president the power to detain suspected terrorists even if they tried to flee battle or conceal themselves as a civilian.
This directly contradicts the International Committee of the Red Cross which believing that detention authority should be limited to ``those who directly participate in hostilities \autocite[639]{bradley2016}'', even though the committee was cited in the Obama Memo.
An additional sign of increasing executive discretion could be found in the aforementioned expansion in the scope of who could be detained under the 2001 AUMF, since it would allow for the president to detain anyone who they deemed as being a supporter of Al Qaeda through the principle of co-belligerency, despite there being no precedent for this type of proceeding in a non-international conflict against decentralized targets.
Although the memo stated that principles of international law would have to inform the interpretations of the 2001 AUMF, it provided no further clarifications on what principles it was referring to or how they would have to inform the legal decisions. 

Despite these factors however, some would still argue that the Obama administration's interpretations of detention law still had some substantial changes from the Bush administration, even if the changes weren't immediate.
Sophia Brill states that the international law arguments were structured in a way that would allow for future changes to the presumably unlawful executive detention system without having to undo the order that was previously created and held throughout the Bush presidency and the early parts of the Obama presidency.

\begin{displayquote}
``The AUMF and international law arguments, in other words, are quintessentially minimalist. They preserve a status quo with regard to particular detainees but contain no broader ambitions for enlarging executive power, asserting an isolationist approach to international law, or reserving the right to ignore Congress. And they are quintessentially realist in they see the policy context of an ongoing effort to clean up a difficult mess as germane to their legal claims about what the AUMF means. In short, the Obama Memo shows an administration trying to maintain some sort of order as it unwinds an institution whose closure was written into an Executive Order on the President’s first full day in office \autocite[48]{brill2010}.''
\end{displayquote}

\noindent
This argument would explain the apparent gap between President Obama's outward policy promise to close Guantanamo Bay and end the unjust executive detentions while internally creating and supporting legal interpretations that allowed the detention center to stay open.
Since the Guantanamo Bay detention center still remains in operation to this day, it is unclear whether or not the changes from the memo have had any substantial effects on the detention policy of the United States government as a whole.

\section*{The Trump Era}
Although this section will be more brief due to there being less research and documentation on the Trump Administration as well as the fact that there weren't significant changes from the Obama Administration in terms of legal interpretations, it is still worth noting the relevance of the 2001 AUMF to the Trump administration's foreign policy.
The Trump campaign in 2016 focused much less on foreign military policy compared to both of the Obama campaigns, but President Trump still seemed to share some of the same rhetoric as Obama relating to getting us out of the endless wars and bringing the troops home as part of his America First proposal.
Simultaneously, President Trump also followed a rhetoric that was highly critical of Obama's national security decisions regarding the threats of ISIS, Al-Qaeda, and ``radical Islamic terrorism'' as a whole.
Despite this, he didn't have many specific policy prescriptions in mind compared to Obama's proposals such as closing down Guantanamo Bay and passing a new authorization of military force.
Additionally, the lack of political experience led to there being more of an emphasis among political scholars on the foreign policy staff of the Trump Administration rather than Donald Trump himself \autocite[24]{starr-deelen2018}.

It seems as if the changes in foreign policy have been more along the lines of application of prior legal interpretations rather than substantial changes in the interpretations themselves.
President Trump has also continued the use of drone strikes outside of the original 9/11 theaters, as was seen in the Obama presidency \autocite[35]{sterio2018}.
Along these lines, the Trump Administration has legally agreed with the prior Bush and Obama administration interpretations of the 2001 AUMF in regards to drone strikes.
The willingness to use lethal force on suspected terrorists, as shown by both the drone strikes as well as specific incidents such as the assassination of Iran's General Qassem Soleimani, demonstrate that Trump has been more than willing to stick to the Bush Administrations armed conflict paradigm instead of attempting to shift approaches like the Obama Administration.
The Trump administration also maintained the covert and secretive nature of executive military force that was found in both of the prior administrations.

\section*{Conclusion}
Throughout the course of this research project, the main thing I learned is how complex the history and discussion around the 2001 AUMF is.
I had no idea how much effort and detail went into the drafting of the resolution, and also how widely it has been applied to the foreign policy of the United States for the last two decades.
I also believe that I have learned a lot about the foreign policy stances of the Bush, Obama, and Trump Administrations in general.
Although I had heard a lot of news and talk about their foreign policy decisions through current events, speeches, and talking with others, it was interesting to see the ways in which the public image of the presidents seemed to in many cases conflict with the actually legal and political actions taken by their administrations.
For example, the common discussion around the Trump presidency might suggest that he has radically shifted our military policy away from that of the Obama Administration (regardless of whether that is a good or bad thing).
Although this was true to an extent, the changes largely seemed to be in his rhetoric and application rather than any kind of legal shift.
This is similar to the talk around the Obama Administration, as it seems that many people don't have an accurate understanding of the actions President Obama regarding executive military power.
While many of his liberal defenders seem to believe that he was able to undo the wrongs of the Bush Administration, many of his opponents on both the left and the right have criticized him for preserving the status quo.
In reality, it seems to be a mix between those stances.
I believe that while President Obama stayed similar in his policy and legal decisions to the President Bush, there were some notable changes in legal frameworks that could lead to positive change in the future.
However, I also believe that it is impossible to overlook the negative sides of Obama's AUMF application whether it be the secretive drone program, the preservation of Guantanamo Bay, or the global expansion of the war on terror.

In regards to what I believe should happen, I am now a firm believer in the stance that the three previous AUMF resolutions ought to be repealed.
This is because of two main reasons.
First, their existence has stretched executive overreach far beyond the power that was imagined by the original authors of the 2001 AUMF resolution.
It is clear from the actions taken by Congress in the drafting process that they wished for a one-time authorization rather than a more general blank check for executive branch military force.
It is also clear that the scope of the authorization was intended to just those who were directly involved in the specific attacks of 9/11, not terrorism as a whole.
I believe that expansions of the AUMFs to include groups such as Al-Shabaab and ISIS is a blatant abuse of power on the side of the Bush, Obama, and Trump administrations.
Second, I believe that their use has negatively affected transparency and accountability within our government.
We saw with the prior presidential administrations that the AUMF was used as a justification for countless instances of covert action, which seems to completely contradict the original meaning of the resolution.
Although the resolutions try to provide oversight through the invocation of the War Power Resolution, it is clear that this oversight has not been achieved as neither Congress nor the American public were informed about the true extent of the war on terror.

Although oversight and limitations could theoretically be implemented through different legal interpretations of the 2001 AUMF rather than completely new legislation, it seems to me that the approach of repealing the resolution would still be preferable.
Due to the \emph{fait accompli} nature of the war on terror that was brought up by Starr-Deelen, I don't see the possibility of future presidents being able to significantly change the legal interpretations of that arose in the Bush and Obama administrations.
As there are currently conflicts that are being passed down from president to president, it is getting to the point where it is much easier for the president to manage the wars rather than completely end them.
Therefore, any kind of change in the future will most likely have to come from either the legislative or judicial branches of our government.
Since the judicial branch is full of appointees from the prior administrations, and because they seemed to largely agree with the interpretations of the Obama administration, there isn't much hope for change in that branch.
As a result, repealing the 2001 AUMF not only seems to be the most preferable approach but also the most realistic approach.

In terms of what is actually being done within Congress right now, there seems to be some kind of willingness to repeal the AUMF resolution, even if it isn't a top priority.
Although there have been many attempts to pass resolutions to repeal the authorizations, they have largely been unsuccessful.
Tim Kaine, among with other Democratic members of Congress, have led the fight against the AUMF with the main opposition being from Republicans.
Senator McConnell has repeatedly blocked resolutions in Senate that would address the AUMFs, and has also spoken about how the United States is not doing enough to combat the rise of ISIS\@.
The closest that Congress has come to a repeal was in 2019, when the House Democrats included a small section in an appropriations bill that would repeal the 2001 AUMF resolution.
Although this did not end up passing through the Senate, it was the first time that any legislation has been passed to repeal an AUMF resolution \autocite{fuller2019}.
There has also been slight support from certain Republicans.
Senator Rand Paul of Kentucky has stated that ``it is indisputably clear that the authority given to the executive under the AUMF passed after 9/11 has become too broad and needs updating \autocite{paul2018}.''
Although it has been almost two decades since the 2001 AUMF was passed, the conversation is far from over.
The time to properly address its problems seems to be getting closer every day.


\newpage
\center{Bibliography}
\printbibliography[heading=none]{}
\end{document}
